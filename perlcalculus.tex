\documentclass{article}
\usepackage[russian]{babel}
\usepackage[utf8]{inputenc}
\usepackage{graphicx}
\newenvironment{excercise}{\paragraph{Упражнение }}{}
\begin{document}
\title{Основы матана с примерами на perl, lisp, Haskell}
\maketitle
\section{Пятиминутка ненависти}
В ней автор распыляется в ненависти к языку Java и C++, которых он не знает и особо не хочет знать.
\section{Полиномы от одной переменной}
Полиномом от одной переменной (или многочленом от одной переменной) называется набор степеней переменной с численными коэффициентами, т.е. формальное выражение вида
$$c_0+c_1*x+c_2*x^2+\cdots+c_n*x^n$$, где $c_0,\dots, c_n$ -- числа, а $x$ -- называется переменной. 
Переменная обозначается какой-нибудь буквой, обычно латинского алфавита. Коэффициенты при некоторой степени переменной могут быть равны нулю, при записи степени с нулевыми коэффициентами опускаются. 

Максимальная степень переменной с ненулевым коэффициентом называется степенью полинома.

Примеры полиномов от одной переменной:
 $1+x+2 x^2+4 x^3$ -- полином 3-й степени, $2$ (число -- это полином нулевой степени), $4y^3$ (полином 3-й степени), $x+5$ (полином 1-й степени), $\frac{1}{2}x+\frac{3}{7}$,
$7 z +5 z^2$. 

Полиномы считаются равными, если  коэффициенты при всех степенях переменной одинаковы. В этом месте есть одна тонкость: можно требовать одинаковости имени переменной у двух полиномов, а можно его не требовать. С точки зрения программной реализации удобнее считать, что имя переменной не важно (в этом случае, мы увидим, результат сложения или умножения двух полиномов всегда однозначно определен). Однако, мы вернемся к этому вопросу когда будем рассматривать полиномы от многих переменных. 
%%note А надо ли требовать одинаковости имени переменной? Для одной переменной это не важно.
\subsection{Представление полиномов в программе}
Поскольку полином однозначно определяется своими коэффициентами, в программе его можно представить массивом или списком коэффициентов.  Для Perl это может быть, например, объект.
\subsection{Вычисление значения полинома}
Схема Горнера.
\subsection{Сложение полиномов}
При сложении полиномов мы приводим подобные слагаемые -- складываем коэффициенты при одной и той же степени переменной. В результате коэффициенты полинома складываются. 
\paragraph{Программная реализация}
\subsection{Перемножение полиномов}
\paragraph{Программная реализация}
\subsection{Замена переменной}
\subsection{График полинома}
Поведение графика на малых значениях аргумента $t$ полинома определяется коэффициентами при самых маленьких степенях $t$. Поведение при больших значениях $t$ определяется старшей степенью $t$. Нарисуем для сравнения $1+10t+20t^2+2t^3+4t^4$, $1+10t$, $4t^4$. 
\paragraph{Отношение полиномов при стремлении аргумента к нулю и бесконечности}
%%%note Это то, что мы будем использовать вместо понятия предела

\section{Мотивация -- приближение функции полиномами}
\section{Бесконечные ряды }
\subsection{Геометрическая прогрессия}
\subsection{Суммирование бесконечных рядов}
\subsection{Радиус сходимости}
Из-за того, что бывают патологические случаи рядов, надо указывать кроме радиуса сходимости минимальный номер элемента, после которого можно пользоваться асимптотической формулой.
\section{Дифференцирование рядов}
Здесь должна быть формула про $f(x)=f(a)+f'(a)(x-a)+\cdots$.
\subsection{Смысл дифференцирования}
\subsubsection{Касательная}
\subsubsection{Физический смысл -- скорость изменения}
\subsubsection{Смысл высших производных}
\subsection{Численные методы дифференцирования}
\section{Интегрирование рядов}
%%note Может быть определить сразу интегрирование как операцию обратную дифференцированию? This will make a story...

Определим интегрирование для ряда вида $$f(x) = \sum a_n x^n$$ вот так
\begin{equation}
\label{eq-integral}
\int dx \sum_{n=0}^{\infty} a_n x^n = C+\sum_{n=0}^{\infty} \frac{a_n}{n+1} x^{n+1} = C+\sum_{n=1}^{\infty}\frac{a_{n-1}}{n}x^n
\end{equation}
Здесь $C$ -- произвольная константа (число). Независимо от ее выбора результирующий ряд будет интегралом от исходного. В программных реализациях можно выбрать $C=0$.

Интегрирование -- операция, обратная дифференцированию. Проверка этого утверждения -- упражнение для читателя.
\begin{excercise}
Доказать что $(\int dx f(x))' = f(x)$ т.е., что
\begin{equation}
\label{eq-int-inverse}
(\int dx  \sum_{n=0}^{\infty} a_n x^n )' =  \sum_{n=0}^{\infty} a_n x^n
\end{equation}
\end{excercise}
\subsection{Смысл интегрирования}
\subsubsection{Площадь под кривой}
В обычных курсах анализа интеграл от функции сразу определяют как
функцию, выражающую площадь под графиком исходной функции. Мы же
покажем это для нашего определения интеграла. Точнее, мы покажем, что
производная от площади под графиком функции есть она сама. 

Схема тут примерно такая: рассмотрим небольшой кусочек площади под
кривой от $t=x$ до $t=x+\delta$. Покажем, что линейная часть
приращения площади при увеличении $t$ от $x$ до $x+\delta$
пропорциональна значению функции в точке $x$ и $\delta$, а все лишнее
пропорционально более высоким степеням $\delta$. Т.о. сама функция
образует линейную часть приращения площади под ней, ergo производная
площади под графиком сама функция, ergo, интеграл, как операция
обратная дифференцированию выражает площадь под графиком.

Посмотрим на картинку \ref{fig-integral}. Покажем, что добавка площади $\delta S(x) = f(x)\delta x + O(\delta x^2)$, а следовательно, по одному из определений производной, производная площади под графиком в точке $x$ равна $f(x)$. Для этого заметим, что $\delta S(x)$ складывается из площади криволинейного треугольника $CFD$ и площади прямоугольника $ABFC$, которая равна $f(x)\delta x$. Оценим площадь треугольника $CFD$ сверху. Она заведомо меньше площади прямоугольника $CFDE$, а его площадь $(f(x+\delta x)-f(x))\delta x$, здесь при достаточно малых $\delta x$  $f(x+\delta x) \simeq f(x)+O(\delta x)$, значит $S_{CFD}<\delta x O(\delta x) = O(\delta x^2)$.   

\begin{figure}
  \includegraphics{integral}
  \caption{\label{fig-integral}Площадь криволинейной трапеции $ABDC$ состоит из площади прямоугольника $ABFC$, которая равна $f(x) \delta x$ и площади криволинейного треугольника $CFD$, которая заведомо меньше площади прямоугольника $CFDE$. }
\end{figure}

\subsubsection{Длина пути при известной скорости}
Собственно, в секции про производные мы определили, что скорость -- это производная от пройденного расстояния по времени. Поскольку интегрирование обратно дифференцированию, то вот и все.
\subsection{Определенные интегралы}
\subsection{Численные методы интегрирования}
Простейшие численные методы интегрирования основаны на том, что интеграл -- площадь под графиком функции
\paragraph{Метод прямоугольников}
\subsection{Методы наивысшей алгебраической точности}
\subsection{Аналитические методы интегрирования}
\subsubsection{Замена переменной}
\section{Элементарные функции}
\section{Функциональные уравнения. Примерное понятие об аналитическом продолжении}
\section{Дифференциальные уравнения}
\subsection{Численные методы решения дифференциальных уравнений}
\subsubsection{Метод Рунге-Кутта 4-го порядка}
\section{Функции от многих переменных}
\subsection{Полиномы от многих переменных}
\subsection{Бесконечные ряды}
\subsection{Частная производная}
\subsubsection{Замена переменных и полная производная}
\section{Функции комплексной переменной}
\subsection{Комплексные числа}
Определение комплексного числа. Сложение и умножение комплексных чисел. Деление комплексных чисел.
\subsection{Полиномы от комплексной переменной}
\subsection{Радиус сходимости бесконечного ряда от комплексной переменной. Особые точки и аналитическое продолжение}
\subsection{Неоднозначность аналитического продолжения. Интегралы по контуру.}

\section{Разные интересные обобщения}
\subsection{Функции от матриц}
\subsection{Кватернионы}
\subsection{Алгебра Клиффорда}
\subsection{Операторы рождения и уничтожения}
\subsection{Алгебры операторов}
\end{document}
