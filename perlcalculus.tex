\documentclass{article}
\usepackage[russian]{babel}
\usepackage[utf8]{inputenc}
\begin{document}
\title{Основы матана с примерами на perl, lisp, Haskell}
\maketitle
\section{Пятиминутка ненависти}
В ней автор распыляется в ненависти к языку Java и C++, которых он не знает и особо не хочет знать.
\section{Полиномы от одной переменной}
Полиномом от одной переменной (или многочленом от одной переменной) называется выражение вида
$$c_0+c_1*x+c_2*x^2+\cdots+c_n*x^n$$, где $c_0,\dots, c_n$ -- числа, а $x$ -- называется переменной.
Примеры полиномов от одной переменной -- $1+x+2 x^2+4 x^3$, $2$, $x+5$, $\frac{1}{2}x+\frac{3}{7}$,
$7 z +5 z^2$. 

Полиномы считаются равными, если они зависят от одной и той же переменной и коэффициенты при всех степенях этой переменной одинаковы.
\subsection{Представление полиномов в программе}
\subsection{Вычисление значения полинома}
Схема Горнера.
\subsection{Сложение полиномов}
При сложении полиномов мы приводим подобные слагаемые -- складываем коэффициенты при одной и той же степени переменной. В результате коэффициенты полинома складываются. 
\paragraph{Программная реализация}
\subsection{График полинома}
Поведение графика на малых значениях аргумента $t$ полинома определяется коэффициентами при самых маленьких степенях $t$. Поведение при больших значениях $t$ определяется старшей степенью $t$. Нарисуем для сравнения $1+10t+20t^2+2t^3+4t^4$, $1+10t$, $4t^4$. 
\section{Мотивация -- приближение функции полиномами}
\section{Бесконечные ряды }
\subsection{Геометрическая прогрессия}
\subsection{Суммирование бесконечных рядов}
\subsection{Радиус сходимости}
Из-за того, что бывают патологические случаи рядов, надо указывать кроме радиуса сходимости минимальный номер элемента, после которого можно пользоваться асимптотической формулой.
\section{Дифференцирование рядов}
Здесь должна быть формула про $f(x)=f(a)+f'(a)(x-a)+\cdots$.
\subsection{Смысл дифференцирования}
\subsubsection{Касательная}
\subsubsection{Физический смысл -- скорость изменения}
\subsubsection{Смысл высших производных}
\subsection{Численные методы дифференцирования}
\section{Интегрирование рядов}
\subsection{Определенные интегралы}
\subsection{Смысл интегрирования}
\subsubsection{Площадь под кривой}
\subsubsection{Длина пути при известной скорости}
\subsection{Численные методы интегрирования}
\subsection{Методы наивысшей алгебраической точности}
\subsection{Аналитические методы интегрирования}
\subsubsection{Замена переменной}
\section{Элементарные функции}
\section{Функциональные уравнения. Примерное понятие об аналитическом продолжении}
\section{Дифференциальные уравнения}
\subsection{Численные методы решения дифференциальных уравнений}
\subsubsection{Метод Рунге-Кутта 4-го порядка}
\section{Функции от многих переменных}
\subsection{Полиномы от многих переменных}
\subsection{Бесконечные ряды}
\subsection{Частная производная}
\subsubsection{Замена переменных и полная производная}
\section{Функции комплексной переменной}
\subsection{Комплексные числа}
Определение комплексного числа. Сложение и умножение комплексных чисел. Деление комплексных чисел.
\subsection{Полиномы от комплексной переменной}
\subsection{Радиус сходимости бесконечного ряда от комплексной переменной. Особые точки и аналитическое продолжение}
\subsection{Неоднозначность аналитического продолжения. Интегралы по контуру.}

\section{Разные интересные обобщения}
\subsection{Функции от матриц}
\subsection{Кватернионы}
\subsection{Алгебра Клиффорда}
\subsection{Операторы рождения и уничтожения}
\subsection{Алгебры операторов}
\end{document}
